\documentclass[a4paper,11pt,english,german]{book}

%%%%%%%%%%%%%%%%%%%%%%%%%%%%%%%%%%%%%%%%%%
%            Packages                    %
%%%%%%%%%%%%%%%%%%%%%%%%%%%%%%%%%%%%%%%%%%
\usepackage[T1]{fontenc} %Set font encoding for output PDF
%\usepackage[utf8]{inputenc} %Important for German "Umlaute"
\usepackage[latin9]{inputenc}
\usepackage{amsfonts,amsmath,amssymb} % AMS and other symbols
\usepackage{graphicx} %Support for including graphics
\usepackage{wrapfig} %To wrap text around figures
\usepackage[english,german]{babel}
\usepackage{textcomp} % Text-Companion-Symbols (e. g. \texteuro)
\usepackage{fancyhdr} %Pretty headers/footers
\usepackage{bm}
\usepackage{blindtext}
\usepackage[numbers]{natbib}
\usepackage{textpos}
\usepackage{url} 
\usepackage{mathtools}
\usepackage{xr} %For correct cross-referencing?
\usepackage{xcolor} %Support for using colours
\definecolor{tum}{rgb}{0.0784,0.4667,0.78824}
\usepackage[unicode=true,bookmarks=true,bookmarksnumbered=true,bookmarksopen=true,bookmarksopenlevel=1,breaklinks=true,pdfborder={0 0 1},backref=false,colorlinks=false]{hyperref}
\hypersetup{pdftitle={THESIS TITLE - THESIS SUBTITLE}, pdfauthor={FIRSTNAME LASTNAME}, pdfsubject={MSc thesis, Heinz-Nixdorf Chair of Biomedical Electronics, Technical University of Munich},pdfkeywords={acoustofluidics, acoustophoresis, ultrasound, microfluidics, fluid dynamics}}

%\usepackage{varioref}
%\usepackage{refstyle}
%\usepackage{booktabs}
%\usepackage{esint}
%\usepackage{array}
%\usepackage{ae} %Fonts for PDF files
%\usepackage{booktabs} %Professional tables
%\usepackage[bf,textfont=small]{caption} %Make pretty captions
%\usepackage[strict]{changepage} %Adjust margins on individual pages
%\usepackage{dcolumn} %Align table columns on decimal point
%\usepackage{latexsym} %AMS and other symbols
%\usepackage{layout} %'\layout' prints margins on page
%\usepackage{listings} %Include code
%\usepackage{lmodern} % Latin Modern font to avoid "pixelation"
%\usepackage{longtable} %Tables that break pages
%\usepackage{lscape} %Rotate pages (text is rotated)
%\usepackage[framed,numbered,autolinebreaks,useliterate]{mcode} %Include matlab code 
%%\usepackage{multicol} %Multicolumned individual pages
%%\usepackage{pdfpages} %Include pdf documents
%\usepackage{pdfsync} %Forward or reverse search
%\usepackage{rotating} %Rotate text
%\usepackage{siunitx} %SI units and numbers
%\usepackage{supertabular} %Tables
%\usepackage{sidecap}
%\epstopdfsetup{update}
%\usepackage{pdflscape} %Rotate pages (page is rotated)
%\usepackage{subfig} %Arrange figures
%\usepackage{keyval} %Used by subfig 
%\usepackage{everysel} %Required by subfig
%\usepackage{ragged2e} %Required by subfig
%\usepackage[final]{listings}
%\usepackage{appendix}
%\usepackage{pdfpages}
%\pdfinclusioncopyfonts=1
%\usepackage{tikz}
%\usepackage{url}
%\usepackage{lmodern}
%\usepackage{multirow}
%\usepackage{makeidx}
%\usepackage{showidx}
%\usepackage{attachfile}
%\usepackage{psfrag}
%\usepackage{lscape}
%\usepackage{longtable}
%\usepackage{BeraMono}
%\usepackage{verbatim}
%\usepackage{xargs}[2008/03/08]
%\PassOptionsToPackage{normalem}{ulem}
%\usepackage{ulem}

%====================================
% PAGE GEOMETRY
%====================================
%\pagestyle{myheadings}
\topmargin       0 mm
\oddsidemargin    10 mm
\evensidemargin   2 mm
\textwidth      145 mm
\textheight     230 mm
\headheight 15 pt

%====================================
% HEADINGS SETUP
%====================================
\pagestyle{fancyplain}
\renewcommand{\chaptermark}[1]{\markboth{#1}{}}
\renewcommand{\sectionmark}[1]{\markright{\thesection\ #1}}
\lhead[\fancyplain{}{\sffamily\bfseries\thepage}]%
      {\fancyplain{}{\sffamily\bfseries\rightmark}}
\rhead[\fancyplain{}{\sffamily\bfseries\leftmark}]%
      {\fancyplain{}{\sffamily\bfseries\thepage}}
\cfoot{}

		
%====================================
% MACROS / NEWCOMMANDS
%====================================
%% EQUATIONS %%
\newcommand{\beq}[1]{\begin{equation} \eqlab{#1}}
\newcommand{\eeq}{\end{equation}}
\newcommand{\bsub}{\begin{subequations}}	
\newcommand{\esub}{\end{subequations}}
\def\bal#1\eal{\begin{align}#1\end{align}}
\def\bsubal#1\esubal{\bsub \begin{align}#1\end{align} \esub}
\newcommand{\nn}{\nonumber}
\newcommand{\notop}{{{}_{}}}

%% REFERENCES %%
\newcommand{\eqlab}[1]{\label{eq:#1}}
\newcommand{\equref}[1]{Eq.~(\ref{eq:#1})}
\newcommand{\eqsref}[2]{Eqs.~(\ref{eq:#1}) and~(\ref{eq:#2})}
\newcommand{\figref}[1]{Fig.~\ref{fig:#1}}
\newcommand{\figsref}[2]{Figs.~\ref{fig:#1} and~\ref{fig:#2}}
\newcommand{\figlab}[1]{\label{fig:#1}}
\newcommand{\appref}[1]{Appendix~\ref{chap:#1}}
\newcommand{\appsref}[2]{Appendices~\ref{chap:#1} and~\ref{chap:#2}}
\newcommand{\chapref}[1]{Chapter~\ref{chap:#1}}
\newcommand{\chapsref}[2]{Chapters~\ref{chap:#1} and~\ref{chap:#2}}
\newcommand{\chaplab}[1]{\label{chap:#1}}
\newcommand{\secref}[1]{Section~\ref{sec:#1}}
\newcommand{\secsref}[2]{Sections~\ref{sec:#1} and~\ref{sec:#2}}
\newcommand{\seclab}[1]{\label{sec:#1}}
\newcommand{\tabref}[1]{Table~\ref{tab:#1}}
\newcommand{\tabsref}[2]{Tables~\ref{tab:#1} and~\ref{tab:#2}}
\newcommand{\tablab}[1]{\label{tab:#1}}

%% LITERAL ABBREVIATIONS %%
\newcommand{\ie}{{i.e.}}
\newcommand{\eg}{{e.g.}}
\newcommand{\etal}{\textit{et~al.~}}
\newcommand{\insitu}{\textit{in~situ}}

%% MATH SYMBOLS %%
\renewcommand{\vec}[1]{\bm{#1}}
\newcommand{\pp}{\partial}
\newcommand{\nablabf}{\boldsymbol{\nabla}}
\newcommand{\ic}{\mathrm{i}} % imaginary unit
\newcommand{\im}{\mathrm{Im}}
\newcommand{\re}{\mathrm{Re}}
\newcommand{\e}[1]{\mathrm e^{#1}}
\newcommand{\avr}[1]{\big\langle #1 \big\rangle}
\newcommand{\ee}{\mathrm{e}}
\newcommand{\ii}{\mathrm{i}}
\newcommand{\dm}{\mathrm{d}}
\newcommand{\ex}{\vec{e}^\notop_x}
\newcommand{\ey}{\vec{e}^\notop_y}
\newcommand{\ez}{\vec{e}^\notop_z}
\newcommand{\rrr}{\vec{r}}
\newcommand{\nx}{n_x}
\newcommand{\ny}{n_y}
\newcommand{\nz}{n_z}

%====================================
% MAIN BODY
%====================================
\graphicspath{{figures/}}
\begin{document}

\thispagestyle{empty}
\frontmatter
\pagenumbering{roman}
\addcontentsline{toc}{chapter}{Preliminaries}

\begin{wrapfigure}{r}{0.15\textwidth}
	\includegraphics[width=0.16\textwidth]{TUM_Logo_blau_cmyk_pdf.pdf}
\end{wrapfigure}
\noindent
\small{\textcolor{tum}{Heinz-Nixdorf-Chair of Biomedical Electronics\\
Department of Electrical and Computer Engineering\\
Technical University of Munich\\} }

\begin{center}

\vspace*{20mm}
\Huge{\bf\sf THESIS TITLE}\\[3mm]
\LARGE{\bf\sf\textit{THESIS SUBTITLE}}
\\[15mm]
{\LARGE Firstname Lastname}
\\[5mm]
{\normalsize MATRIKELNUMMER}
\\[11mm]
INSERT IMAGE
\vspace{\fill}
\\[10mm]
{\large Bachelor thesis, INSERT DATE
\\[3mm]
Supervised by Dr. Rune Barnkob}

\end{center}

\newpage
%
\thispagestyle{empty}
\vspace*{1cm}
{\noindent \textbf{Cover illustration:} Illustration of this and that showing some cool biomedical acoustofluidics applications.}\\[1cm]
{\noindent \textit{THESIS TITLE --- THESIS SUBTITLE}}\\[0.5cm]
%\vfill
{\noindent Copyright \copyright\ 2019 Firstname Lastname All Rights Reserved}\\
{\noindent Typeset using \LaTeX\ and \textsc{Matlab}}\\\\
\vspace*{9cm}
%\vfill
\begin{center}
\large{\textit{MAYBE DEDICATE YOUR WORK}}\\
\large{\textit{TO SOMEONE}}
\end{center}


% 
\chapter*{Abstract}

\addcontentsline{toc}{section}{Abstract}

\blindtext

\selectlanguage{German}%

\chapter*{Zusammenfassung}

\addcontentsline{toc}{section}{Zusammenfassung}

Dies hier ist ein Blindtext zum Testen von Textausgaben. Wer diesen Text liest, ist
selbst schuld. Der Text gibt lediglich den Grauwert der Schrift an. Ist das wirklich so?
Ist es gleichg�ltig, ob ich schreibe: ?Dies ist ein Blindtext? oder ?Huardest gefburn??
Kjift ? mitnichten! Ein Blindtext bietet mir wichtige Informationen. An ihm messe ich
die Lesbarkeit einer Schrift, ihre Anmutung, wie harmonisch die Figuren zueinander
stehen und pr�fe, wie breit oder schmal sie l�uft. Ein Blindtext sollte m�glichst viele
verschiedene Buchstaben enthalten und in der Originalsprache gesetzt sein. Er mu�
keinen Sinn ergeben, sollte aber lesbar sein. Fremdsprachige Texte wie ?Lorem ipsum?
dienen nicht dem eigentlichen Zweck, da sie eine falsche Anmutung vermitteln.

\selectlanguage{english}

\chapter*{Preface}

\addcontentsline{toc}{section}{Preface}

This thesis is submitted as partial fulfillment for obtaining the degree of Doctor of Philosophy
(PhD) at the Technical University of Munich (TUM). The PhD project
was financed by the ...

\vspace{\fill}


\begin{center}
Munich, INSERT DATE
\par\end{center}

\vspace{0.1cm}


\begin{center}
INSERT SIGNATURE
\par\end{center}

\begin{center}
Firstname Lastname
\par\end{center}


\clearpage{}
\addtolength{\parskip}{-0pt}\tableofcontents{}%
\addcontentsline{toc}{section}{Contents}
\markboth{Contents}{Contents}
\clearpage{}
\listoffigures
\addcontentsline{toc}{section}{List of Figures}
\markboth{List of Figures}{List of Figures}
\clearpage{}
\listoftables
\addcontentsline{toc}{section}{List of Tables}
\markboth{List of Tables}{List of Tables}
\clearpage{}
%\chapter*{List of Symbols}
\markboth{LIST OF SYMBOLS}{LIST OF SYMBOLS}
\addcontentsline{toc}{chapter}{List of Symbols}




%% First table/page
\begin{center}
\thicklines
\begin{tabular}{p{2cm}p{8cm}p{3cm}}
Symbol                    	& Description									& Unit \\ \hline\hline
$\equiv$       				& Equal to by definition						& \SI{}{} \\
$\sim$       				& Of the same order								& \SI{}{} \\
$\approx$       			& Approximately equal to						& \SI{}{} \\
$\propto$       			& Proportional to								& \SI{}{} \\
$\ll,\gg$       			& Much smaller than, much greater than			& \SI{}{} \\
$\cdot$       				& Scalar product								& \SI{}{} \\
$\times$       				& Cross-product or multiplication sign			& \SI{}{} \\\\

$\pp_i=\pp/\pp_i$           & Partial derivative with respect to $i$		& $[i]^{-1}$ \\
$\nablabf$                	& Nabla or gradient vector operator				& \SI{m^{-1}}{} \\
$\nablabf\cdot$           	& Divergence vector operator			  		& \SI{m^{-1}}{} \\
$\nablabf\times$          	& Rotation vector operator						& \SI{m^{-1}}{} \\
$\nabla^2$                	& Laplacian scalar operator						& \SI{m^{-2}}{} \\
$\mathcal{O}(x^n)$          & Terms of order $x^n$ and higher powers		& \SI{}{} \\\\

$\langle\circ\rangle$       & Time average of $\circ$						& \SI{}{} \\
$\langle\circ\rangle_{_i}$  & Average of $\circ$ over $i$					& \SI{}{} \\
$|\circ|$       			& Absolute value of $\circ$						& \SI{}{} \\
$(\circ)^*$       			& Complex conjugate of $\circ$					& \SI{}{} \\
$\Delta\circ$       		& A change in $\circ$							& \SI{}{} \\
$\delta\circ$       		& An infinitesimal change in $\circ$			& \SI{}{} \\
$\Real{\circ}$       		& Real part of $\circ$							& \SI{}{} \\
$\Imag{\circ}$       		& Imaginary part of $\circ$						& \SI{}{} \\
$\ic$                     	& Imaginary unit								& \SI{}{} \\
$\e{}$                     	& Euler's constant, $\ln{(\e{})}=1$				& \SI{}{} \\\\

$x$, $y$, $z$               & Rectangular coordinates						& \SI{}{} \\
$\vec{\mathrm{e}}_i$        & Unit vector in $i$-direction					& \SI{}{} \\
$\vec{r}$		   	  		& Position vector								& \SI{m}{} \\
$\vec{n}$		   	  		& Surface normal vector							& \SI{m}{} \\	
$\Omega$		        	& Domain of interest							& \SI{}{} \\
$\dm\Omega$		        	& Boundary of domain $\Omega$					& \SI{}{} \\
\hline
							& \multicolumn{2}{r}{\textit{Continued on next page}}		
\end{tabular}
\end{center}


%% Second table/page
\begin{center}
\begin{tabular}{p{2cm}p{8cm}p{3cm}}
Symbol						& Description             						& Unit \\ \hline\hline
$a_n$		   	  			& Normal acceleration							& \SI{m}{s^{-2}} \\
$\vec{g},g$		   	  		& Gravitational acceleration					& \SI{m}{s^{-2}} \\
$l,L$		 				& Length of channel, length of chip				& \SI{m}{} \\
$w,W$		 				& Width of channel, width of chip				& \SI{m}{} \\
$h,H$		 				& Height of channel, height of chip				& \SI{m}{} \\
$a$		 					& Radius of spherical particle					& \SI{m}{} \\
$R$		 					& Radius of point transducer or fixation pin	& \SI{m}{} \\
$V$		 					& Volume										& \SI{m^3}{} \\
$\ell$		           		& Actuator displacement							& \SI{m}{} \\\\

$\alpha_\mathrm{per}$		& Perturbation parameter				    	& \SI{m}{} \\
$p_i$                     	& $i$'th order acoustic pressure				& \SI{kg}{m^{-1}s^{-2}} \\
$\rho_i$                  	& $i$'th order acoustic mass density  			& \SI{kg}{m^{-3}} \\
$v_i$				      	& $i$'th order acoustic velocity				& \SI{m}{s^{-1}} \\
$\vec{v}_i$		          	& $i$'th order acoustic velocity vector 		& \SI{m}{s^{-1}} \\
$\phi_i$				  	& $i$'th order velocity potential				& \SI{m^2}{s^{-1}} \\\\

$t$			           	  	& Time											& \SI{s}{} \\
$f$			           	  	& Frequency										& \SI{s^{-1}}{} \\
$\omega=2\pi f$			    & Angular frequency								& \SI{s^{-1}}{} \\
$\tau=1/f$			    	& Period										& \SI{s}{} \\\\

$n_i$			  	      	& Number of $\lambda/2$ in spatial direction $i$& \SI{}{} \\
$\lambda$			      	& Acoustic wavelength							& \SI{m}{} \\
$\lambda^*$			      	& Perfect $\lambda/2$-mode wavelength			& \SI{m}{} \\
$\delta$			      	& Relative change in wavelength		    		& \SI{}{} \\
$\alpha$		 			& Aspect ratio parameter				    	& \SI{}{} \\
$\vec{k}$			      	& Complex-valued wave vector					& \SI{m^{-1}}{} \\
$\vec{k}_0$			      	& Real-valued wave vector						& \SI{m^{-1}}{} \\
$\theta$		 			& Dimensionless wavenumber				    	& \SI{}{} \\
$c_a$			          	& Speed of sound in material $a$				& \SI{m}{s^{-1}} \\
$Z_a$			          	& Acoustic impedance of material $a$			& \SI{kg}{m^{-2}s^{-1}} \\
$z$			              	& Acoustic impedance ratio				        & \SI{}{} \\
$p_\mathrm{A}$, $A_j$		& Pressure amplitudes					        & \SI{kg}{m^{-1}s^{-2}} \\

$Q$							& Q-value of acoustic resonance	peak		    & \SI{}{} \\
$\Delta f_i$			    & FWHM of frequency for the $i$'th resonance	& \SI{s^{-1}}{} \\
$\Delta f$			        & Frequency shift between two resonance peaks	& \SI{s^{-1}}{} \\\hline
							& \multicolumn{2}{r}{\textit{Continued on next page}}	
\end{tabular}
\end{center}


%% Second page
\begin{center}
\begin{tabular}{p{2cm}p{8cm}p{3cm}}
Symbol						& Description             						& Unit \\ \hline\hline

$\Eac$						& Time-averaged acoustic energy density 		& \SI{kg}{m^{-1}s^{-2}} \\
$\Espace$					& Spatially- and time-averaged energy density 	& \SI{kg}{m^{-1}s^{-2}} \\
$\tilde{c}$					& Speed of sound ratio							& \SI{}{} \\
$\tilde{\rho}$				& Density ratio							        & \SI{}{} \\
$f_1$, $f_2$			  	& Pre-factors in pressure force expression      & \SI{}{} \\
$U_0$						& Amplitude of acoustic potential	      		& \SI{kg~m^2}{s^{-2}} \\
$\Uac$						& Time-averaged acoustic pressure force potential& \SI{kg~m^2}{s^{-2}} \\ 
$\Fac$						& Time-averaged acoustic pressure force vector  & \SI{kg~m}{s^{-2}} \\
$\Fdrag$					& Stokes drag force vector			       		& \SI{kg~m}{s^{-2}} \\
$\Fg$						& Gravitational force vector		       		& \SI{kg~m}{s^{-2}} \\\\

$\eta$		   	  			& Viscosity										& \SI{kg}{m^{-1}s^{-1}} \\
$\beta\eta$		     	  	& Bulk viscosity								& \SI{kg}{m^{-1}s^{-1}} \\
$\gamma$		   			& Viscous damping factor						& \SI{}{} \\\\

$\Upp$		        		& Driving peak-to-peak voltage					& \SI{kg~m^2}{C^{-1}~s^{-2}} \\
$\Zel$		        		& Electric impedance							& \SI{kg~m^2}{C^{-2}~s^{-1}} \\
$\phiEl$		        	& Electric phase								& \SI{}{} \\\\

$\kB$			           	& Boltzmann's constant							& \SI{kg~m^2}{s^{-2}~K^{-1}} \\
$T$			           	  	& Temperature									& K \\
$D$			           		& Diffusion constant							& \SI{m^2}{s^{-1}} \\
$\ldiff$			        & Diffusion length								& \SI{m}{} \\
\\

$N,M$			    		& Number of measurements						& \SI{}{} \\
$\bar{x}$		    		& Weighted mean of variable $x$					& \SI{}{} \\
$s_x$			    		& Sample standard deviation of variable $x$		& \SI{}{} \\
$\sigma_x^2$			    & Variance of variable $x$						& \SI{}{} \\\\

$\vec{u}$			        & Displacement vector							& \SI{m}{} \\
$\vec{\sigma}$            	& Cauchy stress tensor							& \SI{kg}{m^{-1}s^{-2}} \\
$\lambda_{iklm}$         	& Elasticity tensor								& \SI{kg}{m^{-2}s^{-2}} \\
$\cL$			          	& Longitudinal sound velocity (elastic theory)	& \SI{m}{s^{-1}} \\
$\cT$			          	& Transverse sound velocity (elastic theory)	& \SI{m}{s^{-1}} \\
$\cAv$			          	& Average sound velocity (elastic theory)		& \SI{m}{s^{-1}} \\
$E$			          		& Young's modulus								& \SI{kg}{m^{-1}s^{-2}} \\
$\nu$			          	& Poisson's ratio								& \SI{}{} \\
\hline
\end{tabular}
\end{center}



%\include{listofmaterialparameters}
\mainmatter

\chapter{Introduction}
\chapter{Draft}
\section{Background and remarks}

\section{literature review}


\section{Layer Segmentation}
\subsection{Layer annotation}
sliced mip projection for annotation and then interpolation, backprojected to original volume size

31 volumes are annotated
\subsection{Data preprocessing}
data normalization seems to not matter
different intensity transformations -> choose quadratic intensity transformation with cutoff on both sides
-standard (quadratic)
-enhanced signal
-sliding mip
-> standard seems to work best




\subsection{network}
unet depth 3 and 4: 4 performs much better not in loss but in homogenity

TODO: try unet depth 5 and hopefully it will not be much better but will consume much more memory etc
using dropout in all upconv layer except the "first" one

\subsection{loss functions}
cross entropy loss with linear increased weight away from layer (detection deep in dermis heavy penalty)
precalculation implemented in data loader for performance increase

smoothness loss, smoothness loss works with different parameters, but we choose a low one.
downside: works only in one direction, otherwise dependent on minibatch size

TODO: get a mathematical formulation of the smoothness loss

class weights: light class weight seems to work best (can use class weight to tune the layer thickness)
no class weight (equal) results in rather too thin layer, class weight equal to the inverse of class distribution in the volume leads to rather thick layer

unfortunately a low loss value is not an indicator for a good prediction? but up to now we were always looking at the model
which performs best on the val set, need to compare this to the model trained after 30 epochs

\subsection{cross validation}
no significant loss difference
also slice-wise reshuffling showed no significant loss difference
annotations are maybe not so good

\subsection{post processing}
1. binary fill holes

2. binary closing

3. extract surface data of upper and lower surface (index = value) projection from 3D to 2D

4. apply median filter 26x26, if median[x,y] larger value[x,y], replace value with median. therefore removal of spikes and hills!

5. apply uniform filter 9x5, 9 in direction of minibatch, 5 in direction already smoothened by smoothing loss

6. reconstruct label by filling up the space between the two surfaces.

remarks: tried flipping x and y dimensions and getting two predictions to combine. Network works perfectly in other direction as well, but combination fails. Tried to combine label as well as raw prediction scores, but can't.

\section{Vessel segmentation}
\subsection{synthetic data}

\subsection{vessel annotation}


\subsection{vesnet}



\appendix
\bibliographystyle{ieeetr}
\bibliography{references}

\end{document}
